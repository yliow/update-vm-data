%==============================================================================
% 2024/07/10: instructions for quizzes for ciss240
%==============================================================================
\textsc{Instructions}

In the file \verb!thispreamble.tex! look for 
\begin{console}
\renewcommand\AUTHOR{} 
\end{console}
and enter your email address:
\begin{console}
\renewcommand\AUTHOR{jdoe5@cougars.ccis.edu} 
\end{console}
This is not really necessary since alex will change that for you
when you execute \verb!make!.
Enter your answers in \verb!main.tex!.
In the bash shell, execute \lq\lq \verb!make!" to recompile \verb!main.pdf!.
Execute \lq\lq \verb!make v!" to view \verb!main.pdf!.
Execute \lq\lq \verb!make s!" to create \verb!submit.tar.gz! for submission.
This is also not necessary since alex can create \verb!submit.tar.gz! for you.

You write your answers in the \verb!main.tex! file.
For each question, you'll see boxes for you to fill.
For small boxes, if you see
\begin{console}[frame=single=single,fontsize=\small]
1 + 1 = \answerbox{}.
\end{console}
you do this:
\begin{console}[frame=single=single,fontsize=\small]
1 + 1 = \answerbox{2}.
\end{console}
\verb!answerbox! will also appear in
\lq\lq true/false" and \lq\lq multiple-choice"
questions.

For longer answers that needs typewriter font, if you see
\begin{console}[frame=single=single, fontsize=\small]
Write a C++ statement that declares an integer variable name x.
\begin{answercode}
\end{answercode}
\end{console}
you do this:
\begin{console}[frame=single=single, fontsize=\small]
Write a C++ statement that declares an integer variable name x.
\begin{answercode}
int x;
\end{answercode}
\end{console}
\verb!answercode! will appear in questions asking for
code, algorithm, and program output.
In this case, indentation and spacing is significant.
For program output, I do look at spaces and newlines.

For long answers (not in typewriter font) if you see
\begin{console}[frame=single=single, fontsize=\small]
What is the color of the sky?
\begin{answerlong}
\end{answerlong}
\end{console}
you can write
\begin{console}[frame=single=single, fontsize=\small]
What is the color of the sky?
\begin{answerlong}
The color of the sky is blue.
\end{answerlong}
\end{console}
For students beyond 245: You can put \LaTeX\ commands in
\verb!answerbox! and 
\verb!answerlong!.

A question that begins with \lq\lq T or F or M"
requires you to identify whether it is true or
false, or meaningless.
\lq\lq Meaningless" means something's wrong with the question and
it is not well-defined.
Something like
\lq\lq $1 + 2 = 4$" is either true or false (of course it's false).
Something like
\lq\lq $1 +_2 = 4$?" does not make sense.

When writing results of computations, make sure it's simplified.
For instance write $2$ instead of $1 + 1$.

When writing a counterexample, always write the simplest.
